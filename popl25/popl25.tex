% \documentclass[acmsmall,review,screen,anonymous]{acmart}
\documentclass[acmsmall,review,screen]{acmart}

\acmYear{2025}
\copyrightyear{2025}
\acmJournal{PACMPL}
\acmVolume{9}
\acmNumber{POPL}
\acmArticle{??}
\acmMonth{1}

\expandafter\def\csname @copyrightpermission\endcsname{\raisebox{-1ex}{\includegraphics[height=3.5ex]{cc-by}} This work is licensed under a Creative Commons Attribution 4.0 International License.}
\expandafter\def\csname @copyrightowner\endcsname{\anon{Roblox}.}

\newcommand{\ANY}{\mathtt{any}}
\newcommand{\ERROR}{\mathtt{error}}
\newcommand{\NUMBER}{\mathtt{number}}
\newcommand{\WARN}{\mathsf{Warn}}
\newcommand{\compat}{\sim}

\begin{document}

\title{Gradual Types as Error Suppression}
\subtitle{A Constructive View of Type Warnings}

\author{Lily Brown}
\author{Andy Friesen}
\author{Alan Jeffrey}
\affiliation{
  \institution{Roblox}
  \city{San Mateo}
  \state{CA}
  \country{USA}
}

\begin{abstract}
  This paper presents the view of gradual typing adopted by the
  \anon{Luau} programming language. Prior work on gradual typing has
  been based on \emph{type compatibility}, that is a relation on types
  $T \compat U$ given by contextually closing $T \compat \ANY \compat
  U$.  Type systems based on type compatibility use $\compat$ rather
  than type equivalence (or a similar presentation for languages with
  subtyping). We take a different tack, which is to view type warnings
  \emph{constructively} as a proof object $\WARN(\Gamma \vdash M : T)$
  saying that the type derivation $\Gamma \vdash M : T$ should
  generate a warning. Viewing type warnings constructively allows us
  to talk about \emph{error suppression}, for example the $\ANY$ type
  is error suppressing, and so this type system is gradual in the
  sense that developers can explicitly annotate terms with the $\ANY$
  type to switch off type warnings. This system has the usual ``well-typed
  programs don't go wrong'' result for program which do not have
  explicit type annotations with error-suppressing types, except this
  property can now be stated as the presence of $\WARN(\Gamma \vdash M
  : T)$ rather than the absence of a run-time error. This system has
  been deployed as part of the \anon{Luau} programming language, used
  by \anon{millions of users of Roblox Studio}.
\end{abstract}

\begin{CCSXML}
<ccs2012>
<concept>
<concept_id>10011007.10011006.10011039.10011311</concept_id>
<concept_desc>Software and its engineering~Semantics</concept_desc>
<concept_significance>500</concept_significance>
</concept>
</ccs2012>
\end{CCSXML}
\ccsdesc[500]{Software and its engineering~Semantics}

\maketitle

\section{Introduction}

TODO

\section{Further work}

TODO

\appendix

\section{Pragmatic Semantic Subtyping}

TODO

\bibliographystyle{ACM-Reference-Format}
\bibliography{bibliography}

\end{document}
